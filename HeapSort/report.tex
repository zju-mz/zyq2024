\documentclass[UTF8]{ctexart}
\usepackage{geometry, CJKutf8}
\geometry{margin=1.5cm, vmargin={0pt,1cm}}
\setlength{\topmargin}{-1cm}
\setlength{\paperheight}{29.7cm}
\setlength{\textheight}{25.3cm}

% useful packages.
\usepackage{amsfonts}
\usepackage{amsmath}
\usepackage{amssymb}
\usepackage{amsthm}
\usepackage{enumerate}
\usepackage{graphicx}
\usepackage{multicol}
\usepackage{fancyhdr}
\usepackage{layout}
\usepackage{listings}
\usepackage{float, caption}
\usepackage{booktabs} 

\lstset{
    basicstyle=\ttfamily, basewidth=0.5em
}

% some common command
\newcommand{\dif}{\mathrm{d}}
\newcommand{\avg}[1]{\left\langle #1 \right\rangle}
\newcommand{\difFrac}[2]{\frac{\dif #1}{\dif #2}}
\newcommand{\pdfFrac}[2]{\frac{\partial #1}{\partial #2}}
\newcommand{\OFL}{\mathrm{OFL}}
\newcommand{\UFL}{\mathrm{UFL}}
\newcommand{\fl}{\mathrm{fl}}
\newcommand{\op}{\odot}
\newcommand{\Eabs}{E_{\mathrm{abs}}}
\newcommand{\Erel}{E_{\mathrm{rel}}}

\begin{document}

\pagestyle{fancy}
\fancyhead{}
\lhead{周钰棋, 3230104245}
\chead{数据结构与算法第七次作业}
\rhead{Dec.2th, 2024}

\section{设计思路}

HeapSort.h实现了一个简单的堆排序算法,用于对std::vector中的元素进行排序。整体思路是先利用std::make\_heap将输入元素构建成一个最大堆,然后通过n次调用std::pop\_heap将最大元素逐步移动到向量末尾并且对剩余元素重新构建最大堆,从而实现排序。

\section{测试流程}
首先定义了一系列辅助函数,包括check函数用于检查序列是否有序(升序),以及四个用于生成不同类型测试序列的函数(generateRandomVector、generateSortedVector、generateReverseSortedVector、generatePartiallyDuplicatedVector)。

然后在main函数中,设定了测试序列的长度为1000000,此规模用于模拟实际应用中的大数据场景。接着逐个测试我自定义的堆排序和系统的std::sort\_heap在处理四种数列所用的时间,验证是否排序正确并输出结果。


\begin{table}[h] % "h" 表示将表格放置在当前位置
    \centering % 表格居中
    \caption{效率对比} % 表格标题
    \begin{tabular}{ccc} % "ccc" 表示每列都是居中对齐
        \toprule
        & 自定义堆排序 & std::\text{sort\_heap} time \\
        \midrule
        随机序列 & 0.826334s & 0.675536s \\
        有序序列 & 0.451392s & 0.429709s \\
        逆序序列 & 0.474902s & 0。441454s \\
        部分元素重复序列 & 0.563454s & 0.526029s \\
        \bottomrule
    \end{tabular}
    \label{tab:efficiency_comparison} % 表格标签,用于引用
\end{table}

\section{效率差异原因分析}

在自定义heapsort中,我是先构建了一次最大堆,然后在循环中多次调用pop\_heap完善新的最大堆。而std::sort\_heap在使用前通常也需要先构建堆但可能后续的操作更为简单,如不用先把最大元素移到最后一位而是用更省时的方法。同时标准库的实现可能有更好的协同优化,使得整体效率更高。例如,可能在内存布局、缓存利用等方面进行了优化,减少了数据交换和移动的开销。
\end{document}

%%% Local Variables: 
%%% mode: latex
%%% TeX-master: t
%%% End:

